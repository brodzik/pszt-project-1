\documentclass[12pt,a4paper]{article}
\usepackage[utf8]{inputenc}
\usepackage[T1]{fontenc}
\usepackage[polish]{babel}
\usepackage{float}
\usepackage{cite}
\usepackage{algorithm}
\usepackage{algorithmic}
\usepackage{graphicx}

\begin{document}

\title{[PSZT] RB.S14 A może na wyspę?}
\author{Adrian Brodzik \and Jakub Górka}
\maketitle

\section*{Zadanie}
Porównać standardowy algorytm ewolucyjny z jego wersją, w której zastosowano model wyspowy (ang. \textit{island model}). W modelu tym stosuje się podział populacji na podgrupy. Każda podpopulacja rozwija się oddzielnie, ale co pewien czas występują migracje.

\section*{Teza}
\textbf{Algorytm genetyczny z modelem wyspowym jest szybszy i efektywniejszy niż klasyczny algorytm genetyczny z jedną wyspą.}

\section*{Rozwiązanie}
Model wyspowy działa podobnie jak klasyczny algorytm genetyczny z tą różnicą, że pod koniec każdej generacji może nastąpić migracja osobników do różnych wysp.
\\
\\
Osobniki początkowe są generowane losowo. Wybór $N$ osobników, np. do krzyżowania, jest losowym wyborem ważonym, tzn. prawdopodobieństwo wybrania danego osobnika zależy od jego przystosowania. Nie ma gwarancji, że zostaną wybrani najlepsi. Taka implementacja ma zapewnić zachowanie różnorodności populacji.
\\
\\
Przyjęliśmy, że migracje występują dokładnie co $X$ generacji. Poza tym jest szansa, że emigranci z jednej wyspy trafią do wyspy, z której wyruszyli, czyli migracja może się nie odbyć.
\begin{algorithm}[H]
\caption{Sekwencyjny algorytm genetyczny z modelem wyspowym}
\begin{algorithmic}[1]
	\STATE wygeneruj wyspy
	\STATE
	\WHILE{\NOT koniec}
		\FORALL{wyspy}
			\STATE wybierz dwóch osobników
			\STATE wykonaj krzyżowanie
			\STATE wykonaj mutacje
			\STATE oceń przystosowanie dzieci
			\STATE dodaj dzieci do populacji
			\STATE usuń nieprzystosowane osobniki z populacji
		\ENDFOR
		\STATE
		\IF{czas na migracje}
			\FORALL{wyspy}
			\STATE wybierz osobniki emigrujące
			\STATE wylosuj wyspę docelową
			\STATE zapisz emigrantów i wyspę docelową
			\ENDFOR
			\STATE
			\STATE dodaj emigrantów do odpowiednich populacji wysp
		\ENDIF
	\ENDWHILE
\end{algorithmic}
\end{algorithm}
\noindent
W rzeczywistości nie stosuje się algorytmu sekwencyjnego, tylko algorytm równoległy, który znacznie przyspiesza obliczenia. Każda wyspa dostaje swój proces, na którym tworzone są kolejne generacje populacji. Procesy są synchronizowane wtedy, gdy nadejdzie czas na migracje.
\\
\\
Algorytm został zaimplementowany w języku Python, wykorzystując biblioteki: \texttt{numpy}, \texttt{pandas}, \texttt{tqdm}, \texttt{multiprocessing}, \texttt{math}, \texttt{json}, itp.

\section*{Podział obowiązków}
\begin{table}[H]
	\centering
	\begin{tabular}{l|l}
		\multicolumn{1}{c|}{Adrian Brodzik} & \multicolumn{1}{c}{Jakub Górka} \\ \hline
		implementacja algorytmu & \\
		optymalizacja funkcji Griewanka & optymalizacja funkcji Rastrigina \\
		analiza wstępna & analiza szczegółowa \\
		stylizacja dokumentacji & treść dokumentacji              
	\end{tabular}
\end{table}

\section*{Testowanie}
Jako przykładowe problemy optymalizacyjne wybraliśmy funkcję Rastrigina
\[ f(\mathbf{x})=10n+\sum_{i=1}^{n}\left[x_i^2-10\cos(2\pi x_i)\right] \]
oraz funkcję Griewanka
\[ f(\mathbf{x})=1+\frac{1}{4000}\sum_{i=1}^{n}x_i^2-\prod_{i=1}^{n}\cos\left(\frac{x_i}{\sqrt{i}}\right). \] Obie funkcje posiadają minimum globalne w punkcie $\mathbf{x=0}$ oraz bardzo dużo ekstremów lokalnych. Złożoność (wymiar) problemu ustala parametr $n$.
\\
\\
Wykonaliśmy testy dla funkcji Griewanka dla $n=20$ oraz funkcji Rastrigina dla $n=20$ i $n=40$. Zbadaliśmy kombinacje różnych parametrów:
\begin{itemize}
	\item liczba wysp: $1,2,3,4,5,10,15,20$
	\item liczba osobników jednej wyspy: $100,500,1000$
	\item prawdopodobieństwo mutacji pojedynczego bitu: $10\%,1\%,0.1\%$
	\item częstotliwość migracji (co $X$ generacji): $100,500,1000$
	\item liczba osobników migrujących: $1,2,3,4,5,10$
\end{itemize}
Wyniki testowe można odtworzyć, ustawiając odpowiednie parametry przy uruchamianiu pliku \texttt{main.py}. \textbf{Uwaga}: jeśli liczba wysp wynosi $1$, należy zamienić funkcję \texttt{run\_parallel} na \texttt{run\_single\_island}. Istnieje też możliwość dodania własnej funkcji przystosowania dekodującej genotyp osobników; przykłady znajdują się w \texttt{decoder.py} oraz \texttt{benchmark.py}.
\\
\\
Testowanie zostało przeprowadzone głównie w chmurze serwisu Kaggle. Celem było osiągnąć $przystosowanie=0$. Dla testów z jedną wyspą ograniczenie czasowe wynosiło 9 godzin. Dla testów z więcej niż jedną wyspą ograniczenie czasowe wynosiło od 3 do 5 minut. Każdy problem testowy został zbadany dwa razy (dla dwóch różnych ziaren generatora liczb pseudolosowych). Wszystkie wyniki oraz użyte parametry zostały zapisane i załączone.

\section*{Wyniki}
\begin{table}[H]
	\resizebox{\textwidth}{!}{\begin{tabular}{c|c|c|c|c|c|c}
		\begin{tabular}[c]{@{}c@{}}liczba\\ wysp\end{tabular} & \begin{tabular}[c]{@{}c@{}}liczba\\ osobników\end{tabular} & \begin{tabular}[c]{@{}c@{}}prawd.\\ mutacji\end{tabular} & \begin{tabular}[c]{@{}c@{}}częstotliwość\\ migracji\end{tabular} & \begin{tabular}[c]{@{}c@{}}liczba\\ migrantów\end{tabular} & \begin{tabular}[c]{@{}c@{}}liczba\\ generacji\end{tabular} & przystosowanie \\ \hline
		15 & 100 & 0.001 & 1000 & 1 & 12000 & 0 \\ \hline
		15 & 100 & 0.001 & 1000 & 3 & 10000 & 0 \\ \hline
		15 & 100 & 0.001 & 1000 & 4 & 12000 & 0 \\ \hline
		10 & 100 & 0.01 & 1000 & 10 & 16000 & 0 \\ \hline
		10 & 100 & 0.001 & 1000 & 2 & 11000 & 0 \\ \hline
		15 & 100 & 0.001 & 1000 & 3 & 10000 & 0 \\ \hline
		10 & 100 & 0.001 & 1000 & 4 & 16000 & -0.023 \\ \hline
		15 & 100 & 0.001 & 1000 & 2 & 12000 & -0.024 \\ \hline
		15 & 100 & 0.001 & 1000 & 5 & 12000 & -0.025 \\ \hline
		15 & 100 & 0.001 & 500 & 1 & 11500 & -0.026
	\end{tabular}}
	\caption{Najlepsze wyniki funkcji Griewanka dla $n=20$}
	\label{tab:griewank_20_ranking}
\end{table}
\begin{table}[H]
	\resizebox{\textwidth}{!}{\begin{tabular}{c|c|c|c|c|c|c}
			\begin{tabular}[c]{@{}c@{}}liczba\\ wysp\end{tabular} & \begin{tabular}[c]{@{}c@{}}liczba\\ osobników\end{tabular} & \begin{tabular}[c]{@{}c@{}}prawd.\\ mutacji\end{tabular} & \begin{tabular}[c]{@{}c@{}}częstotliwość\\ migracji\end{tabular} & \begin{tabular}[c]{@{}c@{}}liczba\\ migrantów\end{tabular} & \begin{tabular}[c]{@{}c@{}}liczba\\ generacji\end{tabular} & przystosowanie \\ \hline
			1 & 100 & 0.001 & 0 & 0 & 80279 & -0.394 \\ \hline
			1 & 1000 & 0.001 & 0 & 0 & 49840 & -0.395 \\ \hline
			1 & 1000 & 0.01 & 0 & 0 & 98525 & -0.411 \\ \hline
			1 & 500 & 0.01 & 0 & 0 & 49929 & -0.418 \\ \hline
			1 & 500 & 0.001 & 0 & 0 & 105812 & -0.426
	\end{tabular}}
	\caption{Najlepsze wyniki funkcji Griewanka dla $n=20$ (jedna wyspa)}
	\label{tab:griewank_20_ranking_w1}
\end{table}
\begin{table}[H]
	\resizebox{\textwidth}{!}{\begin{tabular}{c|c|c|c|c|c|c}
			\begin{tabular}[c]{@{}c@{}}liczba\\ wysp\end{tabular} & \begin{tabular}[c]{@{}c@{}}liczba\\ osobników\end{tabular} & \begin{tabular}[c]{@{}c@{}}prawd.\\ mutacji\end{tabular} & \begin{tabular}[c]{@{}c@{}}częstotliwość\\ migracji\end{tabular} & \begin{tabular}[c]{@{}c@{}}liczba\\ migrantów\end{tabular} & \begin{tabular}[c]{@{}c@{}}liczba\\ generacji\end{tabular} & przystosowanie \\ \hline
			10 & 100 & 0.001 & 500 & 1 & 8500 & 0 \\ \hline
			10 & 100 & 0.001 & 500 & 2 & 8000 & 0 \\ \hline
			10 & 100 & 0.001 & 500 & 3 & 7500 & 0 \\ \hline
			10 & 100 & 0.001 & 500 & 4 & 7500 & 0 \\ \hline
			10 & 100 & 0.001 & 500 & 5 & 8000 & 0 \\ \hline
			10 & 100 & 0.001 & 500 & 10 & 8000 & 0 \\ \hline
			10 & 100 & 0.001 & 1000 & 2 & 8000 & 0 \\ \hline
			10 & 100 & 0.001 & 1000 & 3 & 8000 & 0 \\ \hline
			10 & 100 & 0.001 & 1000 & 10 & 8000 & 0 \\ \hline
			2 & 100 & 0.01 & 500 & 2 & 19500 & 0
	\end{tabular}}
	\caption{Najlepsze wyniki funkcji Rastrigina dla $n=20$}
	\label{tab:rastrigin_20_ranking}
\end{table}
\begin{table}[H]
	\resizebox{\textwidth}{!}{\begin{tabular}{c|c|c|c|c|c|c}
			\begin{tabular}[c]{@{}c@{}}liczba\\ wysp\end{tabular} & \begin{tabular}[c]{@{}c@{}}liczba\\ osobników\end{tabular} & \begin{tabular}[c]{@{}c@{}}prawd.\\ mutacji\end{tabular} & \begin{tabular}[c]{@{}c@{}}częstotliwość\\ migracji\end{tabular} & \begin{tabular}[c]{@{}c@{}}liczba\\ migrantów\end{tabular} & \begin{tabular}[c]{@{}c@{}}liczba\\ generacji\end{tabular} & przystosowanie \\ \hline
			1 & 1000 & 0.01 & 0 & 0 & 49822 & -2.168 \\ \hline
			1 & 1000 & 0.001 & 0 & 0 & 33264 & -4.002 \\ \hline
			1 & 1000 & 0.01 & 0 & 0 & 60437 & -4.025 \\ \hline
			1 & 1000 & 0.001 & 0 & 0 & 99927 & -4.975 \\ \hline
			1 & 100 & 0.01 & 0 & 0 & 49934 & -5.969
	\end{tabular}}
	\caption{Najlepsze wyniki funkcji Rastrigina dla $n=20$ (jedna wyspa)}
	\label{tab:rastrigin_20_ranking_w1}
\end{table}
\begin{table}[H]
	\resizebox{\textwidth}{!}{\begin{tabular}{c|c|c|c|c|c|c}
			\begin{tabular}[c]{@{}c@{}}liczba\\ wysp\end{tabular} & \begin{tabular}[c]{@{}c@{}}liczba\\ osobników\end{tabular} & \begin{tabular}[c]{@{}c@{}}prawd.\\ mutacji\end{tabular} & \begin{tabular}[c]{@{}c@{}}częstotliwość\\ migracji\end{tabular} & \begin{tabular}[c]{@{}c@{}}liczba\\ migrantów\end{tabular} & \begin{tabular}[c]{@{}c@{}}liczba\\ generacji\end{tabular} & przystosowanie \\ \hline
			10 & 100 & 0.001 & 1000 & 5 & 8000 & -1.428 \\ \hline
			10 & 100 & 0.001 & 500 & 3 & 7500 & -1.503 \\ \hline
			10 & 100 & 0.001 & 1000 & 10 & 8000 & -1.594 \\ \hline
			10 & 100 & 0.001 & 500 & 1 & 7500 & -1.969 \\ \hline
			10 & 100 & 0.001 & 500 & 2 & 7500 & -2.276 \\ \hline
			10 & 100 & 0.001 & 1000 & 10 & 7000 & -2.642 \\ \hline
			10 & 100 & 0.001 & 1000 & 5 & 7000 & -2.887 \\ \hline
			10 & 100 & 0.001 & 1000 & 3 & 7000 & -2.890 \\ \hline
			10 & 100 & 0.001 & 1000 & 4 & 7000 & -3.192 \\ \hline
			10 & 100 & 0.001 & 1000 & 2 & 7000 & -3.507
	\end{tabular}}
	\caption{Najlepsze wyniki funkcji Rastrigina dla $n=40$}
	\label{tab:rastrigin_40_ranking}
\end{table}
\begin{table}[H]
	\resizebox{\textwidth}{!}{\begin{tabular}{c|c|c|c|c|c|c}
			\begin{tabular}[c]{@{}c@{}}liczba\\ wysp\end{tabular} & \begin{tabular}[c]{@{}c@{}}liczba\\ osobników\end{tabular} & \begin{tabular}[c]{@{}c@{}}prawd.\\ mutacji\end{tabular} & \begin{tabular}[c]{@{}c@{}}częstotliwość\\ migracji\end{tabular} & \begin{tabular}[c]{@{}c@{}}liczba\\ migrantów\end{tabular} & \begin{tabular}[c]{@{}c@{}}liczba\\ generacji\end{tabular} & przystosowanie \\ \hline
			1 & 1000 & 0.01 & 0 & 0 & 82823 & -10.798 \\ \hline
			1 & 500 & 0.001 & 0 & 0 & 78153 & -14.930 \\ \hline
			1 & 1000 & 0.001 & 0 & 0 & 51710 & -15.149 \\ \hline
			1 & 500 & 0.01 & 0 & 0 & 49919 & -15.835 \\ \hline
			1 & 500 & 0.01 & 0 & 0 & 84496 & -15.879
	\end{tabular}}
	\caption{Najlepsze wyniki funkcji Rastrigina dla $n=40$ (jedna wyspa)}
	\label{tab:rastrigin_40_ranking_w1}
\end{table}

\section*{Dyskusja}
Problem o większej złożoności dla modelu z kilkoma wyspami uzyskuje lepsze wyniki przystosowania niż model klasyczny dla problemu o dwukrotnie mniejszej wartości parametru $n$. Oczywiście liczba wysp zależy od rozmiaru przeszukiwanej przestrzeni, np. dla funkcji Rastrigina $n=20$ wystarczyło użycie $10$ wysp.
\\
\\
W modelu klasycznym, pomimo dużej liczby generacji, uzyskane rozwiązanie jest znacznie gorsze od modelu wyspowego, gdzie liczba generacji dla każdej wyspy jest nawet $10$ razy mniejsza. Prawdopodobieństwo mutacji w większości przypadków jest niskie. Mimo że losowe zmiany bitów pozytywnie wpływają na dywersyfikację populacji, to zwiększają ryzyko pogorszenia przystosowania danego osobnika. Rzadkie migracje pozwalają na uzyskanie lepszych wyników, gdyż wyspy potrzebują odpowiedniej liczby iteracji, aby skutecznie przeszukać własną wyspę (przestrzeń lokalną) zanim wyemigrują do innej.

\section*{Wnioski}
Model wyspowy skuteczniej przeszukuje przestrzeń w celu znalezienia optimum niż model klasyczny. Dla jednej wyspy lepsze są większe zbiory populacji, gdyż pozytywnie wpływają na dywersyfikację oraz eksplorację przestrzeni. W przypadku wielu wysp małe populacje były wystarczające, gdyż każda wyspa zajmowała się eksploracją osobno. Im więcej użytych wysp w rozwiązaniu tym lepsza wartość przystosowania dla problemów o większej złożoności w porównaniu do modelu klasycznego. Liczba wysp powinna zależeć od rozmiaru przeszukiwanej przestrzeni. Zbyt duża ilość wysp czy osobników populacji może negatywnie wpłynąć na wyniki.
\\
\\
Realizacji projektu pozwoliła nam dokładnie przeanalizować wpływ poszczególnych parametrów na zachowanie algorytmu genetycznego z modelem wyspowym. Dzięki przeprowadzonym testom dowiedzieliśmy się, że zastosowanie wysp jest lepszym wyborem rozwiązywania problemów o większej złożoności w porównaniu do modelu klasycznego z jedną wyspą.

\nocite{*}
\bibliographystyle{plain}
\bibliography{Dokumentacja}{}

\end{document}